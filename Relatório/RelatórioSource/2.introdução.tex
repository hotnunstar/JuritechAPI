\pagenumbering{arabic}
\vspace{1cm}
\section{Introdução}

\subsection{Contextualização}

\indent \par No âmbito da unidade curricular de Projeto de Desenvolvimento de Software, foi solicitado a elaboração de um projeto, com o tema à nossa escolha.
\indent \par Em grupo, optamos por desenvolver um portal de apoio jurídico.

\subsection{Motivação e Objetivos}

\indent \par O projeto tem como objetivo apresentar os princípios e os valores do desenvolvimento ágil de projetos de software e incentivar os membros da equipa a utilizar as técnicas e ferramentas mais adequadas, à luz destes princípios, ao longo de todo o processo de desenvolvimento do sistema de software.
\indent \par É pretendido que o grupo consiga planear, gerir e executar todas as atividades que constam no processo de desenvolvimento de um sistema de software.

\subsection{Âmbito do projeto}

\indent \par Como mencionado anteriormente, decidimos elaborar um portal de apoio jurídico, onde todos os juristas Portugueses poderão armazenar os processos, relembrar, atualizar, modificar, e planear datas referentes a cada processo que os mesmos terão em mão.
\indent \par A ideia para este projeto surgiu após um stakeholder da área lamentar a falta um site/aplicação que pudesse satisfazer as suas necessidades concretas.
\indent \par Como nenhum membro da equipa tem conhecimento na área, tivemos de proceder a alguma pesquisa e fazer o levantamento de requisitos com a ajuda de um stakeholder que solicitou o desenvolvimento do portal. 
\indent \par Solicitamos, então, uma descrição concreta de todas as componentes funcionais pretendidas, bem como uma síntese geral do assunto a ser implementado, de forma a conseguirmos delinear todas as fases de desenvolvimento do sistema de software.












